\documentclass[10pt,twocolumn,letterpaper]{article}

\usepackage{iccv}
\usepackage{times}
\usepackage{epsfig}
\usepackage{graphicx}
\usepackage{amsmath}
\usepackage{amssymb}

% Include other packages here, before hyperref.

% If you comment hyperref and then uncomment it, you should delete
% egpaper.aux before re-running latex.  (Or just hit 'q' on the first latex
% run, let it finish, and you should be clear).
\usepackage[pagebackref=true,breaklinks=true,letterpaper=true,colorlinks,bookmarks=false]{hyperref}

% \iccvfinalcopy % *** Uncomment this line for the final submission

\def\iccvPaperID{****} % *** Enter the ICCV Paper ID here
\def\httilde{\mbox{\tt\raisebox{-.5ex}{\symbol{126}}}}

% Pages are numbered in submission mode, and unnumbered in camera-ready
\ificcvfinal\pagestyle{empty}\fi
\begin{document}

%%%%%%%%% TITLE
\title{Deep Feedback Neural Network for Weakly Supervised Object Localization}

\author{First Author\\
Institution1\\
Institution1 address\\
{\tt\small firstauthor@i1.org}
% For a paper whose authors are all at the same institution,
% omit the following lines up until the closing ``}''.
% Additional authors and addresses can be added with ``\and'',
% just like the second author.
% To save space, use either the email address or home page, not both
\and
Second Author\\
Institution2\\
First line of institution2 address\\
{\tt\small secondauthor@i2.org}
}

\maketitle
%\thispagestyle{empty}


%%%%%%%%% ABSTRACT
\begin{abstract}
Deep convolutional neural networks have been proven to be a very powerful method in computer vision. In this paper, we will briefly introduce the background of feedbacks in the human visual cortex, which motivates us to develop a computational feedback mechanism in the deep neural networks in the past several weeks. The feedback philosophy helps us to visualize the neural network and understand deeper on how deep neural network works, especially for the deep convolutional neural networks. The feedback framework is also extended to re-train the neural networks to better explore the properties of the natural images to avoid overfitting as well as improve the image recognition accuracy. We show will discuss the plans on future improving the feedback neural network architectures.
\end{abstract}

%%%%%%%%% BODY TEXT
\section{Introduction}

% Start with a little story
\begin{center}
  \fbox{
    \parbox{0.85\linewidth}{
    \noindent
    \emph{``What did you see in this image?''\\
      ``Panda, Tiger, Elephant, Lions''\\
      ``Have you seen the Gorilla?''\\
      ``Oh! I even didn't notice there is a Gorilla.''}
    }
  }
\end{center}

Visual attention typically is dominant by \emph{``goals''} from our mind easily in a top-down manner, especially in the case of object detection in images. Cognitive science explains this in the ``Biased Competition Theory''~\cite{beck2009top,desimone1998visual,desimone1995neural}, that human visual cortex is enhanced by top-down stimuli and non-relevant neurons will be suppressed in feedback loops when searching objects. By ``looking and thinking twice'', both human recognition and detection performances increase significantly especially in images with fuzzy background~\cite{Cichy2014Resolving}. It leads to selectivity in neuron activations~\cite{Kruger2013Deep}, which reduces the chance of recognition being interfered with either noises or distractive patterns.

Inspired by above evidences, we present a novel \emph{Feedback Convolutional Neural Network} architecture in this paper. It achieves this selectivity by joint reasoning outputs of class nodes and activations of hidden layer neurons during the feedback loop. As shown in Figure~\ref{fig:splash0}, during the feedforward stage, the proposed network performs inference from image features in a bottom-up manner as traditional Convolutional Networks; while in feedback loops, it sets up semantic labels (\emph{e.g.}, outputs of class nodes) as the ``goal'' in visual search to infer activation status of hidden layer neurons. The network is powerful to be applied on class model visualization~\cite{simonyan2013deep, zeiler2014visualizing} and object localization even in cluttered scenes with multiple objects.

\setlength{\tabcolsep}{2pt}
\begin{figure}[htb]
\begin{center}
\includegraphics[width=0.95\columnwidth]{figs/splash/splash}
% \vspace{-10pt}
\caption{We propose a novel feedback convnet for weakly supervised object localization in complex scenes with cluttered background. The feedback net is able to utilized both bottom-up image features and top-down semantic labels to infer the hidden layer neuron status to match the localize the corresponding salient area in the image. }
\label{fig:splash0}
\vspace{-10pt}
\end{center}
\end{figure}

\setlength{\tabcolsep}{2pt}
\begin{figure*}[htb]
\begin{center}
\begin{tabular}{ccccccc}
%\rotatebox{90}{\hspace{5mm}Sequential} &
\includegraphics[width=0.13\linewidth]{figs/splash1/original} &
\includegraphics[width=0.13\linewidth]{figs/splash1/panda} &
\includegraphics[width=0.13\linewidth]{figs/splash1/tiger} &
\includegraphics[width=0.13\linewidth]{figs/splash1/gorilla} &
\includegraphics[width=0.13\linewidth]{figs/splash1/lion} &
\includegraphics[width=0.13\linewidth]{figs/splash1/elephant} &
\includegraphics[width=0.13\linewidth]{figs/splash1/localization}\\
{\small (a) Input Image} &
{\small (b) Panda} &
{\small (c) Tiger} &
{\small (d) Gorilla} &
{\small (e) Lion} &
{\small (f) Elephant} &
{\small (g) Localization}
\end{tabular}
% \vspace{-10pt}
\caption{We illustrate the localization power of the feedback net on a multi-object image with cluttered background. (a) shows the original input image which both VggNet and GoogleNet recongize as "comic". (b) - (f) demonstrate the powerfulness of our model understanding the image given paticular object labels. We visualization the gradient of each label w.r.t. image after ther convergence of the feedback neural nets (g) shows the localization power for different objects in this complex image based on the gradient. Note that the weights in the net is obtained from a pre-trained feedforward GoogleNet model for image classification.}
\label{fig:splah}
% \vspace{-30pt}
\end{center}
\end{figure*}

\subsubsection*{Optimization in a Feedback Loop}
% Explain from the machine learning perspective
From a machine learning perspective, the proposed feedback network \emph{adds extra flexibility to Convolutional Networks, to help in capturing visual attention and improving object detection}. Convolutional Neural Network~\cite{lecun1998gradient, Krizhevsky2012ImageNet, Simonyan2014Very} has achieved great success in both machine learning and computer vision recent years. Benefit from large scale of training data, (\emph{e.g.,} ImageNet~\cite{deng2009imagenet}), CNNs are capable to learn filters and image compositions at the same time. Various approaches have been adopted to further increase ability of CNN, by either adding regularization in training~\cite{he2015delving,ioffe2015batch}, or going deeper~\cite{Simonyan2014Very, Szegedy2014Going}. Inspired by Deformable Part-Based Model (DPM)~\cite{Felzenszwalb2010Object}, we utilize a simple yet efficient method to optimize image compositions and assign neuron activations given ``goals'' in visual searching. The algorithm maximizes the posterior response of network given target high-level semantic concepts, in a top-down manner. Compared with traditional bottom-up strategies~\cite{he2015delving, ioffe2015batch} aiming to regularize the network training, the proposed feedback neural network adds flexibilities to perform inference from high-level concepts down to receptive fields.

As the example shown in Figure~\ref{fig:splash0}, given a high-level semantic stimulus ``PANDA'', only the neurons in hidden layers related with the concept ``PANDA'' will be activated by iterative optimization in a feedback loop. As a result, only salient regions related with the concept ``PANDA'' are captured in visualizations. Figure~\ref{fig:splah} also shows the visualizations of saliencies given different semantic concepts for the same input image. As suggested by those results, the feedback network achieves certain level of selectivity and non-relevant suppression during the top-down inference. Comparisons between our method against two other visualization algorithms, Oxford~\cite{simonyan2013deep} and Deconvolution~\cite{zeiler2014visualizing}, are shown in Figure~\ref{fig:visual_compare}. Compared with state-of-the-arts, our feedback framework is capable to allow the model focusing on the most important image areas that improve the class confidence.

\setlength{\tabcolsep}{2pt}
\begin{figure*}
\begin{center}
\begin{tabular}{ccccccccc}
\rotatebox{90}{\hspace{5mm}Oxford} &
\includegraphics[width=0.11\linewidth]{figs/visual_compare/gradient/oxford/panda} &
\includegraphics[width=0.11\linewidth]{figs/visual_compare/saliency/oxford/panda} &
\includegraphics[width=0.11\linewidth]{figs/visual_compare/gradient/oxford/tiger} &
\includegraphics[width=0.11\linewidth]{figs/visual_compare/saliency/oxford/tiger} &
\includegraphics[width=0.11\linewidth]{figs/visual_compare/gradient/oxford/gorilla} &
\includegraphics[width=0.11\linewidth]{figs/visual_compare/saliency/oxford/gorilla} &
\includegraphics[width=0.11\linewidth]{figs/visual_compare/gradient/oxford/lion} &
\includegraphics[width=0.11\linewidth]{figs/visual_compare/saliency/oxford/lion} \\
\rotatebox{90}{\hspace{5mm}Deconv} &
\includegraphics[width=0.11\linewidth]{figs/visual_compare/gradient/deconv/panda} &
\includegraphics[width=0.11\linewidth]{figs/visual_compare/saliency/deconv/panda} &
\includegraphics[width=0.11\linewidth]{figs/visual_compare/gradient/deconv/tiger} &
\includegraphics[width=0.11\linewidth]{figs/visual_compare/saliency/deconv/tiger} &
\includegraphics[width=0.11\linewidth]{figs/visual_compare/gradient/deconv/gorilla} &
\includegraphics[width=0.11\linewidth]{figs/visual_compare/saliency/deconv/gorilla} &
\includegraphics[width=0.11\linewidth]{figs/visual_compare/gradient/deconv/lion} &
\includegraphics[width=0.11\linewidth]{figs/visual_compare/saliency/deconv/lion} \\
\rotatebox{90}{\hspace{5mm}Our} &
\includegraphics[width=0.11\linewidth]{figs/visual_compare/gradient/feedback/panda} &
\includegraphics[width=0.11\linewidth]{figs/visual_compare/saliency/feedback/panda} &
\includegraphics[width=0.11\linewidth]{figs/visual_compare/gradient/feedback/tiger} &
\includegraphics[width=0.11\linewidth]{figs/visual_compare/saliency/feedback/tiger} &
\includegraphics[width=0.11\linewidth]{figs/visual_compare/gradient/feedback/gorilla} &
\includegraphics[width=0.11\linewidth]{figs/visual_compare/saliency/feedback/gorilla} &
\includegraphics[width=0.11\linewidth]{figs/visual_compare/gradient/feedback/lion} &
\includegraphics[width=0.11\linewidth]{figs/visual_compare/saliency/feedback/lion} \\
&
\multicolumn{2}{c}{{\small (a) Panda}} &
\multicolumn{2}{c}{{\small (b) Tiger}} &
\multicolumn{2}{c}{{\small (c) Gorilla}} &
\multicolumn{2}{c}{{\small (d) Lion}} \\
\end{tabular}
% \vspace{-10pt}
\caption{We demonstrate the effectiveness of our method by comparing the class model visualization results against Oxford~\cite{simonyan2013deep} and Deconv~\cite{zeiler2014visualizing}. The input image is the same as Figure 1 (a). We show both the visualization results as well as the saliency map. While both Oxford and Deconv have the same input: the image and an object class label (i.e. tiger, panda, etc.), the gradients computed are often salient on one particular object (i.e. elephant). Our feedback framework allows for the model to focus on the most important image area that improves the class confidence.}
\label{fig:visual_compare}
% \vspace{-30pt}
\end{center}
\end{figure*}

\subsubsection*{Weakly Supervised Object Localization}
% Unify the network: recognition and detection in a single network.
Given the salience maps shown in Figure~\ref{fig:splah} and Figure~\ref{fig:visual_compare}, we further develop a weakly supervised object localization algorithm. Instead of using large amount of supervision (\emph{e.g.}, bounding box positions) in traditional methods such as R-CNN~\cite{girshick2014rich} or using regression model~\cite{erhan2014scalable}, we don't require any localization information in the training stage. In this case, we utilize \emph{a unified network performing both recognition and localization tasks}. Experimental results show that our weakly supervised algorithm using feedback network could achieve similar performance on ImageNet object localization task as GoogLeNet~\cite{Szegedy2014Going} and VGG~\cite{Simonyan2014Very}.

The remainder of this paper is organized as follows: Section~\ref{sec:related_work} introduces the related work, while we formulate our algorithm in Section~\ref{sec:model}. Experiments of visualization and object localization are demonstrated in Section~\ref{sec:experiment}. We conclude this work and future directions in Section~\ref{sec:conclusion}

% Implementation - caffe~\cite{jia2014caffe}

\begin{comment}
Introduction will follow these three aspects:
\begin{description}
  \item[Optimization of model] Using DPM as an inspiration, to claim the model add flexibility and spatial assignments to objects / parts;
  \item[Cognitive] Biased Competition theory. Attention includes bottom-up and top-down. While the latter is driven by high-level sementics;
  \item[Unified Network for recognition and detection]: This is important. In this work, we unify the recognition and attention in a unified network. (But the problem is, how to detect multiple objects in an single image? For example, two dogs, two cats?)
\end{description}
\end{comment}




\begin{comment}

We present a novel feedback neural networks for joint reasoning the class node and hidden layer information. The network is powerful to be applied on model class visualization and object localization even in cluttered scenes with multi objects. The framework is novel and

\textbf{Deep Learning and Deep Convolutional Neural Networks, Feedforward Strcture}
Deep Convolutional networks (ConvNets) have recently enjoyed a great success in large-scale im- age and video recognition (Krizhevsky et al., 2012; Zeiler Fergus, 2013; Sermanet et al., 2014; Simonyan Zisserman, 2014) which has become possible due to the large public image reposito- ries, such as ImageNet (Deng et al., 2009), and high-performance computing systems, such as GPUs or large-scale distributed clusters (Dean et al., 2012). In particular, an important role in the advance of deep visual recognition architectures has been played by the ImageNet Large-Scale Visual Recog- nition Challenge (ILSVRC) (Russakovsky et al., 2014), which has served as a testbed for a few generations of large-scale image classification systems, from high-dimensional shallow feature en- codings (Perronnin et al., 2010) (the winner of ILSVRC-2011) to deep ConvNets (Krizhevsky et al., 2012) (the winner of ILSVRC-2012).

\textbf{Psychological feedback, inference top-down and bottom-up}
While we have outlined in this paper a hierarchical feedfor- ward view on visual processing, it is important to remember that within the visual cortex there are generally more feedback connections than forward connections. Also lateral connec- tions play an important role. This hints at the importance of processes like attention, expectation, top-down reasoning, imagination, and filling in. Many computer vision systems try to work in a purely feed-forward fashion. However, vision is inherently ambiguous and benefits from any prior knowledge available. This may even imply that the knowledge of how the tower of Pisa looks influences the perception of an edge on the level of V1. It also means that a system should be able to produce several hypotheses that are concurrently considered and possibly not resolved [102].

\textbf{This paper Main Contribution}
In this paper, we address the visualisation of deep image classification ConvNets, trained on the large-scale ImageNet challenge dataset [2]. To this end, we make the following three contributions. First, we demonstrate that understandable visualisations of ConvNet classification models can be ob- tained using the numerical optimisation of the input image [5] (Sect. 2). Note, in our case, unlike [5], the net is trained in a supervised manner, so we know which neuron in the final fully-connected clas- sification layer should be maximised to visualise the class of interest (in the unsupervised case, [9] had to use a separate annotated image set to find out the neuron responsible for a particular class). To the best of our knowledge, we are the first to apply the method of [5] to the visualisation of ImageNet classification ConvNets [8]. Second, we propose a method for computing the spatial support of a given class in a given image (image-specific class saliency map) using a single back-propagation pass through a classification ConvNet (Sect. 3). As discussed in Sect. 3.2, such saliency maps can be used for weakly supervised object localisation. Finally, we show in Sect. 4 that the gradient-based visualisation methods generalise the deconvolutional network reconstruction procedure [13].

\textbf{Yurgen's feedback neural networks, Attention neural networks, Deep Boltzman Machines}

\textbf{DPM  Top-down, weakly supervised object detection, localization and parsing}
We describe an object detection system that represents highly variable objects using mixtures of multiscale de- formable part models. These models are trained using a discriminative procedure that only requires bounding boxes for the objects in a set of images. The resulting system is both efficient and accurate, achieving state-of- the-art results on the PASCAL VOC benchmarks [11]– [13] and the INRIA Person dataset [10].
Our approach builds on the pictorial structures frame- work [15], [20]. Pictorial structures represent objects by a collection of parts arranged in a deformable configu- ration. Each part captures local appearance properties of an object while the deformable configuration is charac- terized by spring-like connections between certain pairs of parts.
Detections obtained with a single component person model. The model is defined by a coarse root filter (a), several higher resolution part filters (b) and a spatial model for the location of each part relative to the root (c). The filters specify weights for histogram of oriented gradients features. Their visualization show the positive weights at different orientations. The visualization of the spatial models reflects the “cost” of placing the center of a part at different locations relative to the root.

\textbf{Comparing against Oxford and Deconv}

\textbf{ConvNet Implementation Details}
ConvNet implementation details. Our visualisation experiments were carried out using a single deep ConvNet, trained on the ILSVRC-2013 dataset [2], which includes 1.2M training images, labelled into 1000 classes. Our ConvNet is similar to that of [8] and is implemented using their cuda-convnet toolbox1, although our net is less wide, and we used additional image jittering, based on zeroing-out random parts of an image. Our weight layer configuration is: conv64-conv256- conv256-conv256-conv256-full4096-full4096-full1000, where convN denotes a convolutional layer with N filters, fullM – a fully-connected layer with M outputs. On ILSVRC-2013 validation set, the network achieves the top-1/top-5 classification error of 39.7%/17.7%, which is slightly better than 40.7%/18.2%, reported in [8] for a single ConvNet.

\end{comment}

\section{Related Work}
\label{sec:related_work}

\textbf{Deep Feed-foward Convolutional Neural Networks}

\textbf{Feedback Neural Networks and Attention Models}

\textbf{Deep Boltzman Machine and Deep Belief Networks}

\textbf{Incorporating Top-down for Part Localization}

\textbf{Visualizing and Understanding Neural Networks}

\textbf{Weakly Supervised Object Localization}


\section{Model}
\label{sec:model}

\subsection{Convolutional Neural Networks}

\subsection{Infering the Hidden Neurons, the Discriminative Framework}

\subsection{Class Model Visualization}

\textbf{Relationship to Oxford and Deconv}

\subsection{Object Localization}

\textbf{Relationship to Oxford and Deconv}

\section{Experiments}



\subsection{Datasets}

\subsection{Single Model Visualization on ImageNet}

\subsection{Class Model Visualization on Multiple Object Images}

\subsection{Construction Visualization}

\subsection{Localization}

\begin{table}
\centering
Evaluation on Imagenet 2014 localization test set
\begin{tabular}{|c|c|c|}
\hline
Method & Localization error & Classification error\\
\hline
VggNet-Supervised[] & 25.3 & 7.4 \\
GoogleNet-Supervised [] & 26.4 & 14.8 \\
AlexNet-Weakly[] & & \\
AlexNet-Weakly Feedback & & \\
VggNet-Weakly Feedback & & \\
GooglNet-Weakl Feedback & & \\
\hline
\end{tabular}
\caption{We show the localization evaluation on ImageNet 2014 localization competition. Our model clearly outperforms the weakly supervised approach based on []. Notably, we compare even fairbly well against supervisedly trained localization model, where an extra localization bouding boxes dataset is used. We demonstrate that when learning for classifications objectives, the Deep ConvNet already integrate powerful class specific features for attentioning on the important areas.}
\label{tab:localization_accuracy}
\end{table}

\setlength{\tabcolsep}{2pt}
\begin{figure*}
\begin{center}
\begin{tabular}{cc}
%\rotatebox{90}{\hspace{5mm}Sequential} &
\includegraphics[width=0.45\linewidth]{figs/examples/bic-car1_bike} &
\includegraphics[width=0.45\linewidth]{figs/examples/bic-car1_car} \\
\includegraphics[width=0.45\linewidth]{figs/examples/bic-car2_bike} &
\includegraphics[width=0.45\linewidth]{figs/examples/bic-car2_car} \\
\includegraphics[width=0.45\linewidth]{figs/examples/bic-car3_bike} &
\includegraphics[width=0.45\linewidth]{figs/examples/bic-car3_car} \\ 
{\small (a) Bike} &
{\small (b) Car} \\
\end{tabular}
% \vspace{-10pt}
\caption{We show more qualitative results of our feedback neural network on natual images. For each image, we show the original image gradient w.r.t. the class labels and the updated image gradients w.r.t. class labels after feedback. We select a few images for comparing bike v.s. car, zebra v.s. elephant, dog v.s. cat, indicating that our approach can find the targeted objects given only a trained feedforward convnets and class labels.} 
\label{fig:examples}
% \vspace{-30pt}
\end{center}
\end{figure*}

\section{Conclusion \& Discussion}
\label{sec:conclusion}

We proposes a Feedback Convolusional Neural Networks for class model visualization and object localization.
Our Feedback Neural Networks can infer the hidden neuron status given the bottom level input image and top level class labels.
Experiments on ImageNet localization challengeindicates that our model is superior in weakly supervised object localization, and further experiments demonstrate its powerfulness in distinguishing objects, even under cluttered backgrounds with multiple objects.

(1) Robust
(2) Multi-task
(3)

Figure 1: We show the powerfulness of feedback neural networks on class model visualization and object localizations, even when an image contains very cluttered background and lots of salient objects. Note that we simply use a pertained feedforward multi-class convnets (GoogleNet) model [] trained with ImageNet dataset where each training image only contains one object and no further training is involved. The feedback neural nets are able to adapt its middle-level hidden layers (usually represents object parts) by combining the bottom-up feedforward image features as well as top-down feedback semantic information. (a) input image (b) class model visualization given the class label: panda, elephant, gorilla, tiger, lion, (c) final object localization results based on the class model visualization.


{\small
\bibliographystyle{ieee}
\bibliography{egbib}
}

\end{document}
