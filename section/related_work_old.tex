\section{Related Work}
\label{sec:related_work}

\textbf{Deep Feed-foward Convolutional Neural Networks}
Feedforward multilayer neural networks have achieved good performance in many classification tasks in the past few years, notably achieving the best performance in the ImageNet competition in vision([21] [7]). However, they typically give a fixed outcome for each input image, therefore cannot naturally model the influence of cognitive biases and are difficult to incorporate into a larger cognitive framework. The current frontier of vision research is to go beyond object recognition towards image understanding [16]. Inspired by neuroscience research, we believe that an unified module which integrates feedback predictions and interpretations with information from the world is an important step towards this goal.

\textbf{Feedback Neural Networks and Attention Models}

\textbf{Deep Boltzman Machine and Deep Belief Networks}
Generative models have been a popular approach([5, 13]). They are typically based on a probabilis- tic framework such as Boltzmann Machines and can be stacked into a deep architecture. They have advantages over discriminative models in dealing with object occlusion. In addition, prior knowl- edge can be easily incorporated in generative models in the forms of latent variables. However, despite the mathematical beauty of a probabilistic framework, this class of models currently suffer from the difficulty of generative learning and have been mostly successful in learning small patches of natural images and objects [17, 22, 13]. In addition, inferring the hidden variables from images is a difficult process and many iterations are typically needed for the model to converge[13, 15]. A recent trend is to first train a DBN or DBM model then turn the model into a discriminative network for classification. This allows for fast recognition but the discriminative network loses the generative ability and cannot combine top-down and bottom-up information.

\textbf{Incorporating Top-down for Part Localization}

\textbf{Visualizing and Understanding Neural Networks}
In previous work, Erhan et al. [5] visualised deep models by finding an input image which max- imises the neuron activity of interest by carrying out an optimisation using gradient ascent in the image space. The method was used to visualise the hidden feature layers of unsupervised deep ar- chitectures, such as the Deep Belief Network (DBN) [7], and it was later employed by Le et al.[9] to visualise the class models, captured by a deep unsupervised auto-encoder. Recently, the problem of ConvNet visualisation was addressed by Zeiler et al.[13]. For convolutional layer visualisation, they proposed the Deconvolutional Network (DeconvNet) architecture, which aims to approximately reconstruct the input of each layer from its output.
In this paper, we address the visualisation of deep image classification ConvNets, trained on the large-scale ImageNet challenge dataset [2]. To this end, we make the following three contributions. First, we demonstrate that understandable visualisations of ConvNet classification models can be ob- tained using the numerical optimisation of the input image [5] (Sect. 2). Note, in our case, unlike [5], the net is trained in a supervised manner, so we know which neuron in the final fully-connected clas- sification layer should be maximised to visualise the class of interest (in the unsupervised case, [9] had to use a separate annotated image set to find out the neuron responsible for a particular class). To the best of our knowledge, we are the first to apply the method of [5] to the visualisation of ImageNet classification ConvNets [8]. Second, we propose a method for computing the spatial support of a given class in a given image (image-specific class saliency map) using a single back-propagation pass through a classification ConvNet (Sect. 3). As discussed in Sect. 3.2, such saliency maps can be used for weakly supervised object localisation. Finally, we show in Sect. 4 that the gradient-based visualisation methods generalise the deconvolutional network reconstruction procedure [13].

\textbf{Weakly Supervised Object Localization}

\textbf{Feedback Neural Networks and Attention Neural Networks  Special form of Recurrent Neural Networks}
The dasNet Network. Each image in classified after T passes through the network. After each for- ward propagation through the Maxout net, the output classification vector, the output of the second to last layer, and the averages of all feature maps, are combined into an observation vector that is used by a deterministic policy to choose an action that changes the weights of all the feature maps for the next pass of the same image. After pass T , the output of the Maxout net is finally used to classify the image.
